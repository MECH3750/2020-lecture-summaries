\documentclass[12pt]{article}

\usepackage{amssymb}
\usepackage{amsmath}
\usepackage{graphicx}
\usepackage{comment}
\usepackage{textcomp}
\usepackage{xcolor}

\usepackage{natbib}

\setlength{\parindent}{0pt}
\usepackage[margin=3cm]{geometry}

%\renewcommand{\myvecf}{\underline{\mathbf{f}}}
\newcommand{\myvec}[1]{\underline{\mathbf{#1}}}

%\begin{document}
%\documentclass{article}
%\usepackage[utf8]{inputenc}
%\usepackage{amsmath}

%\title{FEM Example}
%\author{apr }
%\date{May 2018}

%\usepackage{graphicx}


\setlength{\parindent}{0mm}% Removes the indents.
\begin{document}

\begin{center}
{\Huge   MECH3750 PBL Content Summary}

\vspace{6mm}

{\Huge  Week 4}

\end{center}

\vspace{6mm}

{\Large Content:}
{\begin{itemize}
	%\item Even \& odd functions
	\item Fourier Series
	\item Complex Properties
	\item Discrete Fourier Transforms
\end{itemize}}

\vspace{4mm}

{\Large Upcoming assessment:}
{\begin{itemize}
	\item Problem Sheet 4 (due before Week 5 PBL session)
\end{itemize}}

\vspace{4mm}

{Tutors: Nathan Di Vaira, Alex Muirhead, William Snell, Tristan Samson, Jakob Ivanhoe}

%{Tutors: Nathan Di Vaira, Alex Muirhead, William Snell, Tristan Samson, Nicholas Maurer, Jakob Ivanhoe, Robert Watt}

\pagebreak

%\section{Even \& Odd Functions}

\section{Fourier Series}

\vspace{4mm}

A Fourier Series provides a way of approximating any periodic function, i.e., a function which repeats itself infinitely over an interval. 

\vspace{4mm}

It uses the special fact that $sin(x)$ and $cos(x)$ are mutually orthogonal on the interval $[-\pi,\pi]$, and represents the periodic function using an infinite series of these trigonometric functions,

\vspace{2mm}  

$$ f(x) = \frac{a_0}{2} + \sum_{n=1}^\infty a_n cos(nx) + b_n sin(nx). $$

\vspace{4mm}

The general method to find the Fourier Series is to first obtain expressions for the coefficients $a_n$ and $b_n$,

\begin{align*}
a_n &= \frac{1}{\pi} \int_{-\pi}^{\pi} f(x)cos(nx) dx \\[1em]
b_n &= \frac{1}{\pi} \int_{-\pi}^{\pi} f(x)sin(nx) dx \\[1em]
\end{align*}

followed by substitution into the Fourier Series expression up to the desired number of terms.

\vspace{4mm}

An example for finding the Fourier Series of $f(x) = x$ on $[-\pi,\pi]$ is given in the lecture notes.

\newpage

\section{Complex Properties}

\subsection{Complex Exponentials}

\vspace{4mm}

Recall that,

\vspace{2mm}

$$ e^{ix} = cos(x) + isin(x).$$

\vspace{4mm}

\textit{Question 1} requires use of this identity for a number of proofs.

\subsection{Complex Inner Product}

\vspace{4mm}

The inner product for complex vectors is defined slightly differently compared to real vectors.

\vspace{4mm}

For two vectors $\mathbf{p}$ and $\mathbf{q}$ which may contain complex numbers in any of their entries, we take the \textit{conjugate} of the first vector to define the inner product,

\vspace{2mm}

$$ \langle \mathbf{p},\mathbf{q} \rangle = \overline{\mathbf{p}} \cdot \mathbf{q} $$

\vspace{4mm}

Recall that the conjugate of a vector simply changes the sign of all imaginary terms, such that for a vector with two entries,

\vspace{2mm}

\begin{align*}
\mathbf{p} &= (p_1,p_2) = (a + bi, c + di),\\[1em]
\overline{\mathbf{p}} &= (\overline{p_1},\overline{p_2}) = (a - bi, c - di).
\end{align*}

\vspace{4mm}

\textit{Norms} and \textit{distances} are still calculated the same way, but with the conjugate of the first vector.

\newpage

\section{Discrete Fourier Transforms}

\vspace{4mm}

The discrete Fourier transform (DFT) is a numerical analysis tool used to fit a Fourier series to a discrete set of data, rather than a continuous function.

\vspace{4mm}

Similar to the approximations we've been doing thus far, we can approximate a vector of complex or real data,

\vspace{2mm}

$$ \mathbf{f} = (f_0,f_1,...,f_{N-1}), $$

\vspace{4mm}

using a basis set,

\vspace{2mm}

$$ \mathbf{p}_n^{(k)} = \frac{1}{N} exp\left(\frac{ik2\pi n}{N}\right),$$

\vspace{2mm}

%using a basis set,
%
%\vspace{2mm}
%
%$$ \mathbf{p}_n^{(k)} = \frac{e^{ikx_n}}{N},$$
%
%\vspace{2mm}
%
%\begin{center}
% \hspace{2mm} where $x_n = \frac{2\pi n}{N}$, \hspace{1mm} $n=0,1,...,N-1$, \hspace{1mm} $k=0,1,...,N-1 $
%\end{center}
%
%\vspace{4mm}

such that

\vspace{2mm}

$$ \mathbf{y} = a_0 \mathbf{p}^{(0)} + a_1 \mathbf{p}^{(1)} + ... + a_{N-1} \mathbf{p}^{(N-1)}.$$

\vspace{4mm}

This can be thought of as breaking $\mathbf{f}$ down into a summation of different frequencies of index $k$ with weightings $a_k$.

\vspace{4mm}

To find the coefficients $a_k$, we apply least squares such that $\|\mathbf{y}-\mathbf{f}\|^2$ is minimised. This gives a normal set of equations (as seen in the least squares method), and seeing as we have an orthogonal basis set, all off diagonal terms of our design matrix are 0.

\vspace{4mm}

Solving the normal equations, we obtain,

\vspace{2mm}

%$$ a_k = \sum_{n=0}^{N-1} e^{-ikx_n}f_n. $$
$$ a_k = \sum_{n=0}^{N-1} exp\left( -\frac{ik2\pi n}{N} \right)f_n. $$

\vspace{4mm}

Note that, since we have used $N$ vectors to fit $N$ data points, we've achieved an exact representation of our vector $\mathbf{f}$. We can also consider the \textit{inverse} DFT, where $\mathbf{f}$ is recovered from the known $a_n$,

\vspace{2mm}

$$ f_n = \frac{1}{N} \sum_{k=0}^{N-1} a_k exp\left(\frac{ik2\pi n}{N} \right). $$

\end{document}