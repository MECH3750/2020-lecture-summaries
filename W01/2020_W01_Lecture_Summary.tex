\documentclass[12pt]{article}

\usepackage{amssymb}
\usepackage{amsmath}
\usepackage{graphicx}
\usepackage{comment}
\usepackage{textcomp}
\usepackage{xcolor}

\usepackage{natbib}

\setlength{\parindent}{0pt}
\usepackage[margin=3cm]{geometry}

%\renewcommand{\myvecf}{\underline{\mathbf{f}}}
\newcommand{\myvec}[1]{\underline{\mathbf{#1}}}

%\begin{document}
%\documentclass{article}
%\usepackage[utf8]{inputenc}
%\usepackage{amsmath}

%\title{FEM Example}
%\author{apr }
%\date{May 2018}

%\usepackage{graphicx}


\setlength{\parindent}{0mm}% Removes the indents.
\begin{document}

\begin{center}
{\Huge   MECH3750 PBL Content Summary}

\vspace{6mm}

{\Huge  Week 1}

\end{center}

\vspace{6mm}

{\Large Content:}
{\begin{itemize}
	\item Taylor Series
	\item Finite Differences
	\item Newton's Method
\end{itemize}}

\vspace{4mm}

{\Large Upcoming assessment:}
{\begin{itemize}
	\item Problem Sheet 1 \textbf{not} submitted/assessed this week
	\item Quiz 1 (Week 3)
\end{itemize}}

\vspace{4mm}

{Tutors: Nathan Di Vaira, Alex Muirhead, Tristan Samson, Nicholas Maurer, Jakob Ivanhoe}

%{Tutors: Nathan Di Vaira, Alex Muirhead, William Snell, Tristan Samson, Nicholas Maurer, Jakob Ivanhoe, Robert Watt}

\pagebreak

\section{Taylor Series}

Taylor series expansions are used to represent any analytic function, $f(x)$, as a power series about a fixed point $a$,

\vspace{2mm}

$$f(x) = f(a) + \frac{f'(a)}{1!}(x-a) + \frac{f''(a)}{2!}(x-a)^2 + ... + \frac{f^{(n)}(a)}{n!}(x-a)^n + f^{(n+1)}(\zeta)\frac{(x-a)^{n+1}}{(n+1)!},$$

\vspace{4mm}

where the last term is called the remainder, and $a<\zeta<x$. This makes the Taylor series exact.

\vspace{6mm}

It can also be written with higher order terms grouped:

\vspace{6mm}

$$f(x) = f(a) + \frac{f'(a)}{1!}(x-a) + \frac{f''(a)}{2!}(x-a)^2 + H.O.T.$$  

\vspace{6mm}

In this course, Taylor series are especially useful for:
\begin{enumerate}
\item obtaining simplified approximations to analytic functions near a point;
\item obtaining finite difference approximations; and
\item studying the error of finite difference approximations.
\end{enumerate} 

\subsection{Multidimensional Taylor Series}

For a two-dimensional function $f(x,y)$, the Taylor series about the point $(a,b)$ is:

\begin{multline*}
f(x,y) = f(a,b) + f_x(a,b)(x-a) + f_y(a,b)(y-b) + \\ \frac{1}{2}\left[f_{xx}(a,b)(x-a)^2 + 2f_{xy}(a,b)(x-a)(y-b) + f_{yy}(a,b)(y-b)^2\right] + H.O.T $$
\end{multline*}

\vspace{4mm}

Recall $f_x$ and $f_{xx}$ are shorthand notation for the first and second derivatives, respectively.

%This can be generalised to 

\pagebreak

\section{Finite Differences}

Finite difference approximations are used to approximate derivatives of complex functions to which no analytical solution exists.

\vspace{4mm}

The 2nd part of the course exclusively applies these approximations to solve PDEs.

\subsection{First derivatives}

An approximation to the first derivative at a point $x$ can be obtained by considering a finite step $h$. This step can be taken above, below, or to both sides of $x$ to calculate the gradient, yielding the following common finite difference formulas.

\vspace{4mm}

Forward difference (1st order accurate):

\vspace{4mm}

$$u'(x) \approx \frac{u(x+h) - u(x)}{h}$$

\vspace{4mm}

Backward difference (1st order accurate):

\vspace{4mm}

$$u'(x) \approx \frac{u(x) - u(x-h)}{h}$$

\vspace{4mm}

Central difference (2nd order accurate):

\vspace{4mm}

$$u'(x) \approx \frac{u(x+h) - u(x-h)}{2h}$$

\vspace{4mm}

Other approximations can be obtained using the method of undetermined coefficients.

\subsection{Second derivatives}

The common second derivative central difference approximation (2nd order accurate) can be obtained using finite difference approximations of the first derivative.

\vspace{4mm}

$$u''(x) \approx \frac{u(x+h) - 2u(x) + u(x-h)}{h^2}$$

\vspace{4mm}

Taylor series can also be used to obtain the central difference approximation as well as higher order approximations, by taking an expansion of a function's value at $(x+h)$, $(x-h)$, $(x+2h)$, $(x+2h)$, and so on, about a point $x$, and performing algebraic manipulation of the resulting series. 


\pagebreak

\section{Newton's Method}

Newton's method is an iterative approach to solving systems of non-linear functions.

\subsection{Single-variable}

Consider a single-variable function, which we want to find the roots of:

\vspace{2mm}

$$f(x)=0$$

\vspace{2mm}

Taking a linear Taylor series approximation (where $x^{(0)}$ is our first guess):

\vspace{2mm}

$$f(x^{(0)} + h) \approx f(x^{(0)}) + f'(x^{(0)})h$$

\vspace{4mm}

Letting the next approximate solution $x^{(1)}=x^{(0)}+h$ be a root, i.e., $f(x^{(0)}+h)=0$, we obtain:

\vspace{2mm}

\begin{align*}
f'(x^{(0)})h &= -f(x^{(0)}) \\
\\
h &= -\frac{f(x^{(0)})}{f'(x^{(0)})}
\end{align*}

\vspace{4mm}

Therefore, our updated approximate solution is:

\vspace{2mm}

$$x^{(1)} = x^{(0)} + h = x^{(0)} - \frac{f(x^{(0)})}{f'(x^{(0)})}$$

\vspace{4mm}

Repeat iteration of the above update rule until $f(x^{(n)}) \approx 0$.

\subsection{Multi-variable}

For sets of equations with multiple variables $\mathbf{x}=(x_1, x_2, x_3, ...)$,

\vspace{2mm}

$$\boldsymbol{f}(\boldsymbol{x}) = 0$$

$$\boldsymbol{f}'(\boldsymbol{x}) = \nabla \boldsymbol{f}'(\boldsymbol{x}) = J$$

\vspace{4mm}

where,

\[
\mathbf{f}(\mathbf{x}) =
\begin{bmatrix}
f_1(\mathbf{x}) \\[1ex]
f_2(\mathbf{x}) \\[1ex]
\vdots \\
f_m(\mathbf{x})
\end{bmatrix}
\]

and $J$ is the Jacobian:

\[
J =
\begin{bmatrix}
  \frac{\partial f_1}{\partial x_1} & 
    \frac{\partial f_1}{\partial x_2} & 
    \dots \\[1ex] % <-- 1ex more space between rows of matrix
  \frac{\partial f_2}{\partial x_1} & 
    \ddots & 
    \vdots \\[1ex]
  \vdots & 
    \dots & 
    \frac{\partial f_m}{\partial x_m}
\end{bmatrix}
\]

\vspace{4mm}

We then obtain the formula:

\vspace{2mm}

$$\mathbf{f}(\mathbf{x}^{(n)} + \mathbf{h}) \approx \mathbf{f}(\mathbf{x}^{(n)}) + J\mathbf{h}$$

\vspace{4mm}

from which the general update rule is:

\vspace{2mm}

$$ \mathbf{x}^{(n+1)} = \mathbf{x}^{(n)} - J^{-1} \mathbf{f}(\mathbf{x}^{(n)}) $$

\end{document}
