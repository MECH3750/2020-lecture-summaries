\documentclass[12pt]{article}

\usepackage{amssymb}
\usepackage{amsmath}
\usepackage{graphicx}
\usepackage{comment}
\usepackage{textcomp}
\usepackage{xcolor}

\usepackage{natbib}

\setlength{\parindent}{0pt}
\usepackage[margin=3cm]{geometry}

%\renewcommand{\myvecf}{\underline{\mathbf{f}}}
\newcommand{\myvec}[1]{\underline{\mathbf{#1}}}

%\begin{document}
%\documentclass{article}
%\usepackage[utf8]{inputenc}
%\usepackage{amsmath}

%\title{FEM Example}
%\author{apr }
%\date{May 2018}

%\usepackage{graphicx}


\setlength{\parindent}{0mm}% Removes the indents.
\begin{document}

\begin{center}
{\Huge   MECH3750 PBL Content Summary}

\vspace{6mm}

{\Huge  Week 10}

\end{center}

\vspace{6mm}

{\Large Content:}
{\begin{itemize}
	\item Advanced solution techniques for convection
	\begin{itemize}
		\item[--] Higher-order schemes
		\item[--] Non-linear convection
	\end{itemize}
	\item Boundary value problems
\end{itemize}}

\vspace{4mm}

{\Large Upcoming assessment:}
{\begin{itemize}
	\item Problem Sheet 10 (due before Week 11 PBL session)
	\item Assignment II (Due Fri. Week 11)
\end{itemize}}

\vspace{4mm}

{Tutors: Nathan Di Vaira, Alex Muirhead, William Snell, Tristan Samson, Nicholas Maurer, Jakob Ivanhoe, Robert Watt}

%{Tutors: Nathan Di Vaira, Alex Muirhead, William Snell, Tristan Samson, Nicholas Maurer, Jakob Ivanhoe, Robert Watt}

\pagebreak

\section{Advanced solution techniques for convection}
 
\subsection{Higher-order schemes}

Modified finite difference schemes are required to universally achieve second order spatial and temporal convergence, $O((\Delta x)^2)$ and $O((\Delta t)^2)$, for convection problems, while maintaining stability. The Lax-Wendroff scheme uses a second order Taylor series as its basis,

\begin{align*}
u_j^{m+1} &= u_j^m + \Delta t \frac{\partial u_j^m}{\partial t} + \frac{(\Delta t)^2}{2}\frac{\partial^2 u_j^m}{\partial t^2} + H.O.T.\\[1em]
		  &\approx u_j^m - v\Delta t \frac{\partial u_j^m}{\partial x} + v^2\frac{(\Delta t)^2}{2}\frac{\partial^2 u_j^m}{\partial x^2}\\[1em]
		  &\approx u_j^m - v\Delta t \frac{u_{j+1}^m - u_{j-1}^m}{2\Delta x} + v^2\frac{(\Delta t)^2}{2}\frac{u_{j+1}^m - 2u_j^m + u_{j-1}^m}{(\Delta x)^2},\\[1em]
\end{align*}

\vspace{-5mm}

where, rearranging and substituting $\sigma = v \Delta t \Delta x$, we obtain the update equation,

\vspace{2mm}

$$ u_j^{m+1} \approx \frac{\sigma}{2}(1+\sigma)u_{j-1}^m + (1-\sigma^2)u_j^m - \frac{\sigma}{2}(1-\sigma)u_{j+1}^m. $$

\vspace{4mm}

The MacCormack Method, which is based on an explicit predictor-correct finite difference scheme, presents a more general method for solving hyperbolic systems, both linear and non-linear.



\subsection{Non-linear convection}

Non-linear convection arises in the inviscid euler equations, which are described by the inviscid Burgers Equation,

\vspace{2mm}

$$ \frac{\partial u}{\partial t} + u \frac{\partial u}{\partial x} = 0, $$

\vspace{4mm}

where the convective velocity is now dependent on the convected quantity, $u$. These can be written in its non-conservative (above) and conservative form,

\vspace{2mm}

$$ \frac{\partial u}{\partial t} + \frac{1}{2} \frac{\partial u^2}{\partial x} = 0. $$

\vspace{4mm}

This conservative form is useful when shocks in solution can form, at which point PDEs are no longer applicable.


\section{Boundary value problems}

The most simple boundary value problems (BVP) are based on one-dimensional, linear ODEs,

\vspace{2mm}

$$ \frac{d^2u}{dx^2} + a\frac{du}{dx} + bu = 0.$$

\vspace{4mm}

To solve the BVP, we need the values of $u$ at each end of our solution domain $[0,L]$,

\vspace{2mm}

$$ u(0)=U_0, $$\\[-4ex]
$$ u(L)=U_L.$$ 

\vspace{4mm}

These types of BVPs are typically for steady-state problems, such as trajectories or temperature distributions (today's PBL sheet),

\vspace{2mm}

$$ \frac{d^2T}{dx^2} + q(T) = 0.$$

\vspace{4mm}

Note that this is the 1D Laplace (elliptical) equation, and that adding a temporal term would result in a diffusion equation,

$$  \frac{\partial T}{\partial t} + \frac{d^2T}{dx^2} + q(T) = 0.$$






\end{document}