\documentclass[12pt]{article}

\usepackage{amssymb}
\usepackage{amsmath}
\usepackage{graphicx}
\usepackage{comment}
\usepackage{textcomp}
\usepackage{xcolor}

\usepackage{natbib}

\setlength{\parindent}{0pt}
\usepackage[margin=3cm]{geometry}

%\renewcommand{\myvecf}{\underline{\mathbf{f}}}
\newcommand{\myvec}[1]{\underline{\mathbf{#1}}}

%\begin{document}
%\documentclass{article}
%\usepackage[utf8]{inputenc}
%\usepackage{amsmath}

%\title{FEM Example}
%\author{apr }
%\date{May 2018}

%\usepackage{graphicx}


\setlength{\parindent}{0mm}% Removes the indents.
\begin{document}

\begin{center}
{\Huge   MECH3750 PBL Content Summary}

\vspace{6mm}

{\Huge  Week 3}

\end{center}

\vspace{6mm}

{\Large Content:}
{\begin{itemize}
	\item Inner Products, Norms, Distances and Orthogonality
	\item Least Squares using Inner Products
\end{itemize}}

\vspace{4mm}

{\Large Upcoming assessment:}
{\begin{itemize}
	\item Problem Sheet 3 (due before Week 4 PBL session)
\end{itemize}}

\vspace{4mm}

{Tutors: Nathan Di Vaira, Alex Muirhead, Nicholas Maurer, Jakob Ivanhoe, Robert Watt}

%{Tutors: Nathan Di Vaira, Alex Muirhead, William Snell, Tristan Samson, Nicholas Maurer, Jakob Ivanhoe, Robert Watt}

\pagebreak

\section{Inner Products}

\vspace{4mm}

The inner product, $ \langle p,q \rangle $, is an operation between two quantities (e.g., functions or vectors) which produces a scalar quantity.

\vspace{4mm}

It can be thought of as a generalisation of the dot product for vectors,

\vspace{2mm}

$$ \langle \mathbf{p},\mathbf{q} \rangle = \mathbf{p} \cdot \mathbf{q}, $$

\vspace{4mm}

and can be defined for functions as,

\vspace{2mm}

$$ \langle p,q \rangle = \int_a^b p(x)q(x) dx. $$

\vspace{4mm}

\subsection{Norms, Distances and Orthogonality}

\vspace{4mm}

These are important properties of the inner product.

\vspace{6mm}

{\renewcommand{\arraystretch}{1.5}

\begin{tabular}{p{40mm}p{15mm}p{50mm}}
%\begin{tabular}{l l l l}
Norm (magnitude) & $\Rightarrow$	& $\|\mathbf{p}\| = \sqrt{\langle \mathbf{p},\mathbf{p} \rangle}$ \\[1em]
Distance & $\Rightarrow$			& $\|\mathbf{p}-\mathbf{q}\| = \sqrt{\langle \mathbf{p}-\mathbf{q},\mathbf{p}-\mathbf{q} \rangle}$ \\[1em]
Orthogonality & $\Rightarrow$	& $\langle \mathbf{p},\mathbf{q} \rangle = 0$ \\[1em]
\end{tabular}

\vspace{4mm}

For the inner product for functions shown above, these properties become,

\vspace{6mm}

\begin{tabular}{p{40mm}p{15mm}p{70mm}}
Norm & $\Rightarrow$			& $\|p\| = \sqrt{\int_a^b p(x)^2 dx}$ \\[1em]
Distance & $\Rightarrow$		& $\|p-q\| = \sqrt{\int_a^b \left( p(x) - q(x) \right)^2 dx}$ \\[1em]
Orthogonality & $\Rightarrow$	& $\langle p,q \rangle = \int_a^b p(x)q(x) dx = 0$ \\[1em]
\end{tabular}

\newpage

\section{Least Squares using Inner Products}

\vspace{4mm}

The least squares matrix equations from last week can be written in a generalised form using the inner product (known as the normal equations),

\vspace{4mm}

\begin{align*}
\begin{bmatrix}
	\langle f,p_1 \rangle \\[1ex]
	\langle f,p_2 \rangle \\[1ex]
	\vdots \\[1ex]
	\langle f,p_n \rangle \\[1ex]
\end{bmatrix} 
&=
\begin{bmatrix}
  	\langle p_1,p_1 \rangle & \langle p_1,p_2 \rangle & \dots & \langle p_1,p_n \rangle \\[1ex]
    \langle p_2,p_1 \rangle & \ddots & \ddots & \vdots \\[1ex]
    \vdots & \ddots & \ddots & \vdots \\[1ex]
    \langle p_n,p_1 \rangle & \dots & \dots & \langle p_n,p_n \rangle \\
\end{bmatrix}
\begin{bmatrix}
	\alpha_1 \\[1ex]
	\alpha_2 \\[1ex]
	\vdots \\[1ex]
	\alpha_n \\[1ex]
\end{bmatrix} 
 \\[2ex]
\end{align*}

The normal equations can now be used to approximate any function or vector, $f$, with a set of other functions or vectors, $p_i$, where the approximation is $y=\sum_{i=1}^{n} \alpha_i p_i$ (PBL \textit{Questions 1 \& 2}).

\vspace{4mm}

In other words, for the least squared problem, we are minimising $\|\mathbf{y}-\mathbf{f}\|^2$.

\vspace{4mm}

If the approximating functions or vectors are mutually orthogonal, the above matrix is diagonal.

\subsection{Orthogonal Polynomials}

\vspace{4mm}

There exist sets of mutually orthogonal polynomials for which the normal equation matrix is diagonal. One such set is the shifted Legendre Polynomials,

\begin{align*}
P_0(x) &= 1 \\[0.5em]
P_1(x) &= 2x - 1 \\[0.5em]
P_2(x) &= 6x^2 - 6x + 1 \\[0.5em]
P_3(x) &= 20x^3 - 30x^2 + 12x - 1 \\[0.5em]
\end{align*}

\end{document}