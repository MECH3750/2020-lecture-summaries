\documentclass[12pt]{article}

\usepackage{amssymb}
\usepackage{amsmath}
\usepackage{graphicx}
\usepackage{comment}
\usepackage{textcomp}
\usepackage{xcolor}

\usepackage{natbib}

\setlength{\parindent}{0pt}
\usepackage[margin=3cm]{geometry}

%\renewcommand{\myvecf}{\underline{\mathbf{f}}}
\newcommand{\myvec}[1]{\underline{\mathbf{#1}}}

%\begin{document}
%\documentclass{article}
%\usepackage[utf8]{inputenc}
%\usepackage{amsmath}

%\title{FEM Example}
%\author{apr }
%\date{May 2018}

%\usepackage{graphicx}


\setlength{\parindent}{0mm}% Removes the indents.
\begin{document}

\begin{center}
{\Huge   MECH3750 PBL Content Summary}

\vspace{6mm}

{\Huge  Week 6}

\end{center}

\vspace{6mm}

{\Large Content:}
{\begin{itemize}
	\item Separation of Variables (Fourier's Method)
\end{itemize}}

\vspace{4mm}

{\Large Upcoming assessment:}
{\begin{itemize}
	\item Problem Sheet 6 (due before Week 7 PBL session)
	\item Assignment 1 (Due Fri. Week 7)
\end{itemize}}

\vspace{4mm}

{Tutors: Nathan Di Vaira, William Snell, Tristan Samson, Nicholas Maurer, Robert Watt}

%{Tutors: Nathan Di Vaira, Alex Muirhead, William Snell, Tristan Samson, Nicholas Maurer, Jakob Ivanhoe, Robert Watt}

\pagebreak


\section{Separation of Variables (Fourier's Method)}

The method of separation of variables assumes a PDE is \textit{separable}, i.e., it can be written in the form, 

$$u(x,t) = F(x)g(t).$$

\subsection{Wave Equation}

Suppose we have a system (e.g. string occupying $0 \leq x \leq L$) which is described by the wave equation,

\begin{align*}
\frac{\partial^2u}{\partial t^2} &= c^2 \frac{\partial^2u}{\partial x^2}  \\[0.5em]
u_{tt} &= c^2u_{xx}
\end{align*}

with initial conditions,

\begin{align*}
u(x,0) &= f(x), \\[0.5em]
u_t(x,0) &= g(x),
\end{align*}

and boundary conditions,

\begin{align*}
u(0,t) &= 0, \\[0.5em]
u(L,t) &= 0.
\end{align*}

\vspace{2mm}

To solve the PDE, firstly substitute the separated equation, $u(x,t) = F(x)g(t)$, into the boundary conditions,

\begin{align*}
u(0,t)=0=F(0)g(t), \\[0.7em]
u(L,t)=0=F(L)g(t), 
\end{align*}

\vspace{4mm}

to obtain the useful results $F(0)=0$ and $F(L)=0$.

\vspace{4mm}

Next, substitute the separated equation into the original PDE,

\begin{align*}
u_{tt} &= c^2u_{xx},  \\[0.8em]
F(x)g''(t) &= c^2F''(x)g(t),
\end{align*}

\vspace{2mm}

to obtain the result,

$$ \frac{g''(t)}{g(t)c^2} = \frac{F''(t)}{F(t)} = k. $$

\vspace{4mm}

In the lectures, it was shown that the solution to this equation is,

\vspace{2mm}

\begin{center}
\begin{tabular}{p{70mm}p{40mm}}
$F(x) = ax+b$ 	& for $k=0$, \\[1em]
$F(x) = a\hspace{0.2em}cosh(\mu x) + b\hspace{0.2em}sinh(\mu x)$ 	& for $k=\mu^2$, \\[1em]
$F(x) = a\hspace{0.2em}cos(p x) + b\hspace{0.2em}sin(p x)$ 	& for $k=-p^2$. \\[1em]
\end{tabular}
\end{center}

These results provide the starting point for PBL \textit{Question 1}. 

\vspace{4mm}

By applying the boundary conditions, it was shown that the case $k=-p^2$ provides the only useful solution, $p=n\pi/L$, such that

\vspace{2mm}

$$ F(x) = b\hspace{0.2em}sin\left(\frac{n\pi}{L}x\right), $$

and,

$$ g(t) = a\hspace{0.2em}cos\left(\frac{n\pi}{L}ct\right) + b\hspace{0.2em}sin\left(\frac{n\pi}{L}ct\right), $$

\vspace{4mm}

where $c=x/t$. Substituting these results into the separated equation, $u(x,t) = F(x)g(t)$, we obtain,

\vspace{2mm}

$$u(x,t) = \left[A_n\hspace{0.2em}cos\left(\frac{n\pi}{L}ct\right) + B_n\hspace{0.2em}sin\left(\frac{n\pi}{L}ct\right)\right]\hspace{0.2em}sin\left(\frac{n\pi}{L}x\right). $$

\vspace{4mm}

Seeing as any value of $n$ gives a solution to the original PDE, and any linear combination of the equations for any $n$ is also a solution, the general solution is,

\vspace{2mm}

$$u(x,t) = \sum_{n=1}^{\infty}\left[A_n\hspace{0.2em}cos\left(\frac{n\pi}{L}ct\right) + B_n\hspace{0.2em}sin\left(\frac{n\pi}{L}ct\right)\right]\hspace{0.2em}sin\left(\frac{n\pi}{L}x\right). $$

\vspace{4mm}

Like our previous work with Fourier series, the summation can now be thought of as breaking the solution to the PDE into a combination of trigonometric functions of different frequencies.

\vspace{4mm}

Imposing the initial conditions ($f(x)=u(x,0)$ and $g(x)=u_t(x,0)$)  expressions for the coefficients were found in lectures,

\vspace{2mm}

$$ A_n = \frac{2}{L} \int_0^L sin\left(\frac{n\pi x}{L}\right) f(x) dx,  $$
\vspace{2mm}
$$ B_n = \frac{2}{n\pi c} \int_0^L sin\left(\frac{n\pi x}{L}\right) g(x) dx.  $$

\vspace{4mm}

Separation of variables may also be applied to the 1D heat equation and steady state heat conduction in 2D, as derived in lectures.

\end{document}