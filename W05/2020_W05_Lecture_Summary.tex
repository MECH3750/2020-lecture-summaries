\documentclass[12pt]{article}

\usepackage{amssymb}
\usepackage{amsmath}
\usepackage{graphicx}
\usepackage{comment}
\usepackage{textcomp}
\usepackage{xcolor}

\usepackage{natbib}

\setlength{\parindent}{0pt}
\usepackage[margin=3cm]{geometry}

%\renewcommand{\myvecf}{\underline{\mathbf{f}}}
\newcommand{\myvec}[1]{\underline{\mathbf{#1}}}

%\begin{document}
%\documentclass{article}
%\usepackage[utf8]{inputenc}
%\usepackage{amsmath}

%\title{FEM Example}
%\author{apr }
%\date{May 2018}

%\usepackage{graphicx}


\setlength{\parindent}{0mm}% Removes the indents.
\begin{document}

\begin{center}
{\Huge   MECH3750 PBL Content Summary}

\vspace{6mm}

{\Huge  Week 5}

\end{center}

\vspace{6mm}

{\Large Content:}
{\begin{itemize}
	\item Introduction to Partial Differential Equations
	\item Initial Conditions \& Boundary Conditions 
\end{itemize}}

\vspace{4mm}

{\Large Upcoming assessment:}
{\begin{itemize}
	\item Problem Sheet 5 (due before Week 6 PBL session)
	\item Quiz 2 (Week 6)
	\item Assignment 1 (Due Fri Week 7)
\end{itemize}}

\vspace{4mm}

{Tutors: Nathan Di Vaira, Alex Muirhead, William Snell, Tristan Samson, Robert Watt}

%{Tutors: Nathan Di Vaira, Alex Muirhead, William Snell, Tristan Samson, Nicholas Maurer, Jakob Ivanhoe, Robert Watt}

\pagebreak


\section{Introduction to Partial Differential Equations (PDEs)}

\vspace{4mm}

In lectures we were introduced to the three core PDEs.

\vspace{4mm}

\hspace{8mm} Wave equation:

\begin{align*}
\frac{\partial^2u}{\partial t^2} &= c^2 \frac{\partial^2u}{\partial x^2}  \\[0.8em]
u_{tt} &= c^2u_{xx}
\end{align*}

\hspace{8mm} Heat (diffusion) equation:

\begin{align*}
\frac{\partial u}{\partial t} &= k \frac{\partial^2u}{\partial x^2}  \\[0.8em]
u_{t} &= k^2u_{xx}
\end{align*}

\hspace{8mm} Laplace's equation:

\begin{align*}
\frac{\partial^2u}{\partial x^2} + \frac{\partial^2u}{\partial y^2} &= 0  \\[0.8em]
u_{xx} + u_{yy} &= 0
\end{align*}

\vspace{4mm}

$u$ may represent any quantity, such as a displacement (wave equation), a concentration or a temperature (heat/diffusion equation).

\vspace{4mm}

These PDEs are written here in their lowest dimensional forms, but may be extended to higher spatial dimensions ($y,z$), in which case the Laplace operator is commonly used,

\vspace{2mm}

$$ \bigtriangledown^2 = \frac{\partial^2u}{\partial x^2} + \frac{\partial^2u}{\partial y^2} + \frac{\partial^2u}{\partial z^2}. $$

\newpage

\subsection{Initial Conditions}

In order to solve a PDE as we progress forward through time, we need to know what it originally looked like, i.e., at $t=0$.

\vspace{4mm}

For the wave equation, we must know the original position and velocity,

\vspace{2mm}

$$ u(x,0), \\[1em]$$
$$ \frac{\partial u}{\partial t}(x,0). $$

\vspace{4mm}

For the heat equation, only initial concentration is required,

\vspace{2mm}

$$ u(x,0) = f(x). $$

\vspace{4mm}

Sometimes we may be required to solve for initial conditions given the solution to $u(x,t)$ (PBL \textit{Question 3}).

\subsection{Boundary Conditions}

Like initial conditions, we must also know what the boundaries limiting the motion of our system are in order to solve a PDE.

\vspace{4mm}

For the wave equation, we typically know (or solve) the value of $u$ at the beginning and end of the spatial interval $[0,L]$ (PBL \textit{Question 3}),

\vspace{2mm}

$$ u(0,t), $$
$$ u(L,t). $$

\vspace{4mm}

These are an example of fixed, or \textit{Dirichlet}, boundary conditions. The other common boundary condition is the \textit{Neumann} boundary condition, 

\vspace{2mm}

$$ \frac{\partial u}{\partial x}(L,x) = f(x), $$

\vspace{4mm}

which effectively specifies a flow, or flux, over a boundary. This is common to diffusion problems. We'll come back to \textit{Dirichlet} and \textit{Neumann} boundary conditions as we progress through solving PDEs throughout the rest of the course.

\vspace{4mm}

Note that for ODEs where time is the only variable, $y(t)$, a boundary value problem requires solving $y(t_1)$ and $y(t_2)$ (PBL \textit{Questions 1 \& 2}).

\end{document}